\documentclass[12pt,a4paper]{article}
\usepackage[top=1in, bottom=1in, left=1in, right=1in]{geometry}
\usepackage{graphicx,setspace,hyperref,mdwlist,comment}
% mini table of contents
\usepackage{minitoc}
\dosecttoc % make section toc
\setcounter{secttocdepth}{2} % subsection depth
\renewcommand{\stctitle}{} % no title
\nostcpagenumbers

\setlength{\marginparwidth}{.5in}
\usepackage{natbib}
% Two lines to create in-text full citations for a syllabus
% And comment out my other standard bibtex commands
\usepackage{bibentry}
\newcommand{\reading}[2][]{\noindent --{#1} \bibentry{#2}.\vspace{.25em}\\}
\newcommand{\seealso}{\noindent \emph{See Also:}\\}
\newcommand{\topic}[1]{\noindent \textbf{#1}\\}
\usepackage[T1]{fontenc}
\usepackage{lmodern}
\hypersetup{
    bookmarks=true,         % show bookmarks bar?
    unicode=false,          % non-Latin characters in Acrobat’s bookmarks
    pdftoolbar=true,        % show Acrobat’s toolbar?
    pdfmenubar=true,        % show Acrobat’s menu?
    pdffitwindow=false,     % window fit to page when opened
    pdfstartview={FitH},    % fits the width of the page to the window
    pdftitle={Syllabus: Issues in American Politics and Government},    % title
    pdfauthor={Thomas J. Leeper},     % author
    pdfsubject={Political Science},   % subject of the document
    pdfnewwindow=true,      % links in new window
    pdfborder={0 0 0}
}

\title{Issues in American Politics and Government }
\author{Thomas J. Leeper\\
Department of Political Science and Government\\
Aarhus University}

\begin{document}
\nobibliography*

\maketitle

\faketableofcontents

%\section{Introduction}

The United States is a unique political system. It is one of the longest-running democracies in the world, has a relatively rare presidential system, is populated by a broad mix of racial, ethnic, religious, economic, and cultural groups, and takes an aggressive, frequently independent, role in other countries' affairs. This seminar dives into several important aspects of American democracy and politics to understand what shapes political activity in the United States. Students will leave the course with a deep understanding of the institutional, historical, philosophical, and cultural factors that shape American politics and will be able to better analyze policymaking and political events in the United States as a result. Broadly the course asks students to consider why things are the way they are in the United States and why things happen the way they do. In addressing these questions, the emphasis is placed on answering the questions `who has power in the United States?' and `what do they do with it?'


\section{Objectives}
The learning objectives for the course are as follows. By the end of the course, students should be able to:

\begin{enumerate}
\item Identify and explain dominant themes that shape (and have shaped) the dynamics of American politics from the founding to the present 
\item Describe political polarization in the contemporary United States, as well as its origins and political effects 
\item Describe political and economic inequalities in the United States and their consequences for political activity and policymaking 
\item Explain institutional roles and functions of branches of the federal government, states, citizens, media, parties, and other political actors 
\item Discuss the roles and power of citizens in American government and policymaking 
\item Apply knowledge of United States political history and political science theories to understand contemporary political events 
\item Evaluate activities of American political institutions and citizens, including their {\em de jure} powers and {\em de facto} operations 
\end{enumerate}

\section{Exam}
Students will be evaluated via an oral examination with a written synopsis based upon issues raised in the course. In preparation for the exam, students are expected to participate in weekly group presentations. These presentations will cover the week's reading material and involve leading a discussion on that material.

\section{Reading Material}
The assigned material for the course consists of empirical research articles and book chapters, all of which are available online or in the printed compendium. There is no textbook.

\clearpage
\section{Schedule}
The general schedule for the course is as follows. Note that our first class meeting is in Week 37. Details on the readings for each week are provided on the following pages.

\secttoc

\clearpage


\subsection{No class (Week 36)}

\clearpage
\subsection{American Values and Opinions (Week 37)}
\emph{What ideals and values define American politics and society? What do Americans value and how do those values shape their political opinions?}
\vspace{1em}

\subsubsection*{Readings}
\reading[Selections from ]{Tocqueville2000}
\reading{Hartz1991} % Hartz in Classic Readings volume
\reading{Smith1993} % multiple traditions
\reading{MacKuenEriksonStimson1989}
\reading{ConoverFeldman1981}
\reading{SullivanPiersonMarcus1979}
\reading[Ch. 1--2 (pp. 7--36) from ]{Friedman1962} % Milton Friedman, Capitalism and Freedom

\seealso
\reading{Feldman1988}
\reading{PageShapiro1992}
% Manifest Destiny



\clearpage
\subsection{The American Founding (Week 38)}
\emph{What does the American Constitution of 1787 say? How does it distribute rights and powers among the branches of national government, states, and citizens? What controversies did the constitution create and resolve? How have those challenges been subsequently addressed?}

\vspace{1em}
\subsubsection*{Readings}
{\em On independence from Britain}:\\
-- The Virginia Declaration of Rights (1776). Available at: \url{http://www.archives.gov/exhibits/charters/virginia_declaration_of_rights.html}\\
-- The Declaration of Independence (1776)\\
-- Paine, Thomas. (1776). Common Sense: Thoughts on the Present State of American Affairs. Available at: \url{http://www.gutenberg.org/files/3755/3755-h/3755-h.htm#chap3}\\
-- The Articles of Confederation (1778)\\

\noindent {\em On the American Constitution}:\\
-- The Constitution of the United States of America (1787)\\
\reading[Ch.1 (27--49) from ]{Maier2010}
\reading[1--2, 10, 45, 65, 69, 70 from ]{FederalistPapers}
-- Letters of Brutus II. (1787). Available from: \url{http://www.constitution.org/afp/brutus02.htm}\\% Letters of Brutus

\seealso
-- Winthrop, John. 1630. A Model of Christian Charity.\\
\reading{Riker1988}
\reading{Riker1996}
\reading{Dahl2006}
\reading{Skowronek1982}
\reading{Skocpol1995}



\clearpage
\subsection{Congress (Week 39)}
\emph{What are the formal and informal institutions of Congress? How are Congressional decisions made and who influences those decisions?}
\vspace{1em}
\subsubsection*{Readings}
\reading{SchicklerWawro2011} % filibuster
\reading{Polsby1968}
\reading{DenzauMunger1986}
\reading{Kollman1997}
\reading{HallWayman1990} % Moneyed interests
\reading{EdwardsBarrettPeake1997} % The Legislative Impact of Divided Government
\reading{SnyderGroseclose2000} % “Party Influence and Congressional Roll-Call Voting.” 
\reading{Erikson1978}
\reading[Ch.6 (129--159) from ]{HibbingTheiss-Morse2002}


\seealso
\reading{ShepsleWeingast1994}
\reading{CoxMcCubbins1993}
\reading{Krehbiel1993} % Where’s the Party?
\reading{Krehbiel1998}
\reading{Austen-SmithWright1994} % Counteractive Lobbying
\reading{Stratmann2002} % Stratmann, Thomas. 2002. “Can Special Interests Buy Congressional Votes? Evidence from Financial Services Legislation.” Journal of Law and Economics 45: 345-373.
\reading{Krehbiel2000} % “Party Discipline and Measures of Partisanship.”
\reading{DiermeierVlaicu2011}
\reading{MillerStokes1963}
\reading{Mayhew1974}
\reading{Fenno1978}






\clearpage
\subsection{The Presidency and Executive Branch (Week 40)}
\emph{What powers does the President have? And can the President do with those powers? How does the President respond to and influence American politics?}
% Frontline: Cheney's War

\vspace{1em}
\subsubsection*{Readings}
\reading{RagsdaleTheis1997}
\reading[Selections from ]{Neustadt1960}
\reading[Selections from ]{Cameron2000}
\reading{Canes-Wrone2001}
\reading{Cohen1995} % Presidential Rhetoric and the Public Agenda
\reading{Moe1987}
\reading{Cooper2005} % signing statements

\seealso
% Terry Moe, “Political Control and the Power of the Agent,” Journal of Law, Economics and Politics, 2005. 21(1): 1-29.
\reading{WhittingtonCarpenter2003}
\reading{JacobsShapiro1995a}
\reading{JacobsShapiro1995b}
\reading{BaumKernell1999}
\reading{McCubbinsSchwartz1984}



\clearpage
\subsection{Courts and Judicial Decision Making (Week 41)}
\emph{What role do courts, and especially the Supreme Court, have in the American political process? How do courts influence policy?}

\vspace{1em}
\subsubsection*{Readings}
\reading{HowardSegal2002} % originalism
\reading{KnightEpstein1996} % “The norm of stare decisis”
\reading{MondakSmithey1997} % public opinion
\reading{GibsonCaldeiraSpence2003} % Bush v. Gore and legitimacy
\reading{BinderMaltzman2002} % Senate delays in judicial appointments
\reading{HuberGordon2004} % elections and judicial punitiveness

\seealso
\reading{Dahl1957} % Journal of Public Law
\reading{McCloskey2000}
\reading{JohnsonMartin1998} % public opinion in response to court decisions
\reading{SegalSpaeth1996} % “The influence of stare decisis on the votes of United States Supreme Court justices”
% McCulloch v. Maryland
\reading{Kersch2002} % “The Reconstruction of Constitutional Privacy Rights and the New American State”
% Jeffrey A. Segal,“Separation of powers games in the positive theory of law and the courts,” American Political Science Review, vol. 91 (1997), pp. 28–44.
\reading{Hutchings2001} % Clarence Thomas vote
\reading{Epsteinetal2006} % "The Changing Dynamics of Senate Voting on Supreme Court Nominees"
% Justin Crowe, “The Forging of Judicial Autonomy,” Journal of Politics 69 (February 2007): 73-87.






\clearpage
\subsection{No class (Week 42)}

\clearpage
\subsection{State and Local Governments (Week 43)}
\emph{How does politics work in state and local governments in the United States? How do states interact with one another and the national government?}

\vspace{1em}
\subsubsection*{Readings}
% state politics
\reading{Volden2006} % States as Policy Laboratories: Emulating Success in the Children’s Health Insurance Program
\reading{MeinkeHasecke2003} % “Term Limits, Professionalization, and Partisan Control in U.S. State Legislatures.”
\reading{EriksonWrightMcIver1989} % state opinion-policy congruence
\reading{Rigby2012} % Obamacare resistance
% city/local politics
\reading{SchaffnerStrebWright2001} % turnout and partisan ballot
% something about racial contact
\reading{OliverHa2007} % Vote Choice in Suburban Elections

\seealso
\reading[Selections from ]{Dahl1961} % Who Governs?
% King, James D. 2000. “Changes in Professionalism in U.S. State Legislatures.” Legislative Studies Quarterly 25:327-343. 
% something on nullification? http://en.wikisource.org/wiki/South_Carolina_Exposition_and_Protest
\reading{ReganDeering2009} % REAL ID resistance
\reading{WrightSchaffner2002} % The Influence of Party: Evidence from the State Legislatures
\reading{LaxPhillips2011} % The Democratic Deficit in the States
\reading{HajnalGerberLouch2002} % Minorities and Direct Legislation: Evidence from California Ballot Proposition Elections
\reading{GerberHopkins2011} % “When Mayors Matter: Estimating the Impact of Mayoral Partisanship on City Policy.”




\clearpage
\subsection{Partisan Politics (Week 44)}
\emph{How do parties compete in the United States? What is the nature of American partisanship and how does partisanship influence politics?}
\vspace{1em}

\subsubsection*{Readings}
% Lipset: Why no socialism in America?; Eric Foner 1984
\reading{APSA1950b} % APSA Report
\reading{KogerMasketNoel2010}
\reading{HiranoSnyder2007} % decline of third parties
\reading[Ch. 6--7 (120--167) from ]{Campbelletal1960}
\reading{MacKuenEriksonStimson1989}
\reading{IyengarSoodLelkes2012} % Affective Polarization
\reading{Fiorina1980} % Daedalus

\seealso
\reading{Aldrich1995}
\reading{Bawnetal2012} % Theory of Parties
\reading{GreenPalmquistSchickler2004}








\clearpage
\subsection{Campaigning and Elections (Week 45)}
\emph{How do US elections work? Who votes? What actors influence election outcomes?}
% electoral college
% 1950 Party Platforms and 2012 party platforms
% Look at election results for 2014

\vspace{1em}
\subsubsection*{Readings}
\reading{Burdenetal2014}
\reading{LauRovner2009}
\reading{Tufte1975} % midterms
\reading{EnosFowlerVavreck2014}
\reading{Gelmanetal2007}
\reading{McDonald2004} % Comparative Analysis of Redistricting Institutions
\reading{Kang2010} % Citizens United v. Federal Election Commission (2010)

\seealso
\reading{Key1955} % or Burnham
\reading{Mayhew2000}
\reading{HuddyTerkildsen1993} % Gender stereotypes
\reading{FridkinKenneyWoodall2008}
\reading{Brooks2013} % He Runs, She Runs
\reading{Geer2006} % In Defense of Negativity
\reading{AnsolabehereIyengar1997}
\reading{McDonaldPopkin2001}
\reading{Gerberetal2011}
\reading{GerberGreen2000}
\reading{Hillygus2005}
\reading{Nielsen2012} % Ground Wars






\clearpage
\subsection{Elite and Mass Polarization (Week 46)}
\emph{How politically polarized is the United States? Where does polarization come from and what are its effects?}

\vspace{1em}
\subsubsection*{Readings}
\reading{APSA1950b} % APSA Report
\reading{FiorinaAbrams2008}
\reading{Levendusky2009b}
% something on primaries?
\reading{Hetherington2001}
\reading{DruckmanPetersonSlothuus2013}
\reading{McCartyPooleRosenthal2009} % gerrymandering
\reading{Mutz2002}

\seealso
\reading{Bishop2008} % The Big Sort
\reading{PooleRosenthal1984}
\reading{Abramowitz2008} % primaries
\reading{MutzReeves2005}
\reading{ArceneauxJohnsonMurphy2012}






\clearpage
\subsection{Inequality (Week 47)}
\emph{How much inequality is there in the United States and what forms does inequality take? What impacts does inequality have on American society and politics? What do Americans think about inequality?}
\vspace{1em}
\subsubsection*{Readings}
-- APSA Task Force on Inequality and American Democracy. (2006). American Democracy in an Age of Rising Inequality. Washington, DC: American Political Science Association. Available from: \url{http://www.apsanet.org/imgtest/taskforcereport.pdf}\\
\reading[Ch ??? from ]{Bartels2008}
\reading{PageBartelsSeawright2013}
\reading{PageShapiro1983}
\reading{VerbaBurnsSchlozman2003} % Unequal at the Starting Line
% Something about the welfare state?

\seealso
\reading{Domhoff1967}
\reading{WintersPage2009}
\reading{Sumner2003} Available from: \url{http://www.gutenberg.org/ebooks/18603}
\reading{Friedman1962}







\clearpage
\subsection{Politics of Race (Week 48)}
\emph{In what ways to race influence political beliefs, attitudes, and behaviors? How are policies racialized and race politicized?}
% Brown v. Board of Education (1954)

\vspace{1em}
\subsubsection*{Readings}
\reading{HutchingsValentino2004}
\reading[Volume I, Chapter 10 (316--407) from ]{Tocqueville2000}
\reading{Gilens1996} % “‘Race Coding’ and White Opposition to Welfare.”
\reading{Kuklinskietal1997} % “Racial Prejudice and Attitudes toward Affirmative Action,”
\reading{SearsHenslerSpeer1979} % White Opposition to Busing
\reading{CameronEpsteinOHalloran1996} % “Do Majority-Minority Districts Maximize Black Substantive Representation in Congress?”
\reading{WilliamsCollins2001} % racial disparities in health
\reading{PettitWestern2004} % incarceration


\seealso
\reading{Takaki1993} % Different Mirror
\reading{KatznelsonMettler2008}
\reading{Mendelberg2010}
\reading{KinderSanders1996}
\reading{Mansbridge1999} % “Should Blacks Represent Blacks and Women Represent Women? A Contingent ‘Yes,’”
\reading{ButlerBroockman2011}
\reading{ValentinoBraderJardina2013}
\reading{CohenDawson1993}
\reading{Fitzhugh1857} % Cannibals All?
\reading{McAdam1982}
\reading{Chong1991} % civil rights movement
% Kinder vs. Sniderman









\clearpage
\subsection{Economic and Foreign Policy (Week 49)}
\emph{What do Americans think of the welfare state, foreign policy, and immigration? Why does the United States have the policies that it has in these domains?}

% Terrorism: Hamdi v. Rumsfeld/Hamdan v. Rumsfeld
% Torture Memos/John Yoo
% Addington/Executive privilege

\vspace{1em}
\subsubsection*{Readings}
% economic policy
\reading{Hacker1998} % health policy
\reading{Bartels2005} % Homer gets a tax cut
\reading{Kuklinskietal2000} % welfare information
% foreign policy
\reading{Gainesetal2007} % partisan interpretation of Iraq
\reading{JacobsPage2005}
% immigration policy
\reading{Muste2013}
\reading{CitrinGreenMusteWong1997}

\seealso
% Graham Allison, “Conceptual Models and the Cuban Missile Crisis,” American Political Science Review, 1969. 63(3): 689-718.
% GroupThink
% Analogical reasoning
\reading{Mettler2011} % submerged states
\reading{Gilens2009} % Gilens - Why Americans Hate Welfare
\reading{Soss1999}






\clearpage
\subsection{Social Policy and the ``Culture Wars'' (Week 50)}
\emph{Why are issues like abortion, homosexuality, guns, and school prayer persistently politicized and contentious? What do Americans think about these issues, and how do those opinions shape party behavior and vote choices? What explains policy in these domains?}
\vspace{1em}
\subsubsection*{Readings}
\reading{MulliganGrantBennett2013} % cultural issues

% District of Columbia v. Heller (2008)
% Lawrence v. Texas (2003)

% gay rights
\reading{LaxPhillips2009}
\reading{Brewer2008}

% death penalty
\reading{NaginPepper2012} % NAP Report: DETERRENCE AND THE DEATH PENALTY

% United States vs. Windsor (2013) and Scalia’s Dissent

\seealso
\reading{Adams1997} % abortion
\reading{BaumgartnerDeBoefBodstun2008} % death penalty
\reading{PeffleyHurwitz2007} % death penalty



\clearpage
\subsection{Wrap-up (Week 51)}
\emph{What have we learned? What implications does it have for understanding American politics and politics elsewhere?}
\vspace{1em}
\subsubsection*{Readings}
\reading{Noel2010} % The Forum
\reading[Ch.2 (62--98) from ]{DelliCarpiniKeeter1997}

% load bibtex, but don't generate a bibliography
\bibliographystyle{plain}
\nobibliography{Syllabus}

\end{document}
