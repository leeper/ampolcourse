\documentclass[12pt,a4paper]{article}
\usepackage[top=1in, bottom=1in, left=1in, right=1in]{geometry}
\usepackage{graphicx,setspace,hyperref,amsmath,amsfonts,multirow,ccaption,mdwlist,comment}
% mini table of contents
\usepackage{minitoc}
\dosecttoc % make section toc
\setcounter{secttocdepth}{2} % subsection depth
\renewcommand{\stctitle}{} % no title
\nostcpagenumbers

% optionally include commented environments
\excludecomment{lessonplan}

\setlength{\marginparwidth}{.5in}
\usepackage{natbib}
% Two lines to create in-text full citations for a syllabus
% And comment out my other standard bibtex commands
\usepackage{bibentry}
\newcommand{\reading}[2][]{\noindent --{#1} from \bibentry{#2}.\vspace{.25em}\\}
\newcommand{\seealso}{\noindent \emph{See Also:}\\}
\newcommand{\topic}[1]{\noindent \textbf{#1}\\}
\usepackage[T1]{fontenc}
\usepackage{lmodern}
\hypersetup{
    bookmarks=true,         % show bookmarks bar?
    unicode=false,          % non-Latin characters in Acrobat’s bookmarks
    pdftoolbar=true,        % show Acrobat’s toolbar?
    pdfmenubar=true,        % show Acrobat’s menu?
    pdffitwindow=false,     % window fit to page when opened
    pdfstartview={FitH},    % fits the width of the page to the window
    pdftitle={Syllabus: Issues in American Politics and Government},    % title
    pdfauthor={Thomas J. Leeper},     % author
    pdfsubject={Political Science},   % subject of the document
    pdfnewwindow=true,      % links in new window
    pdfborder={0 0 0}
}

\title{Issues in American Politics and Government }
\author{Thomas J. Leeper\\
Department of Political Science and Government\\
Aarhus University}

\begin{document}
\nobibliography*

\maketitle

\faketableofcontents

%\section{Introduction}

The United States is a unique political system. It is one of the longest-running democracies in the world, has a relatively rare presidential system, is populated by a broad mix of racial, ethnic, religious, economic, and cultural groups, and takes an aggressive, frequently independent, role in other countries' affairs. This seminar dives into several important aspects of American democracy and politics to understand what shapes political activity in the United States. Students will leave the course with a deep understanding of the institutional, historical, philosophical, and cultural factors that shape American politics and will be able to better analyze policymaking and political events in the United States as a result. Broadly the course asks students to consider why things are the way they are in the United States and why things happen the way they do. In addressing these questions, the emphasis is placed on answering the questions `who has power in the United States?' and `what do they do with it?'


\section{Objectives}
The learning objectives for the course are as follows. By the end of the course, students should be able to:

\begin{enumerate}
\item Identify and explain dominant themes that shape (and have shaped) the dynamics of American politics from the founding to the present 
\item Describe political polarization in the contemporary United States, as well as its origins and political effects 
\item Describe political and economic inequalities in the United States and their consequences for political activity and policymaking 
\item Explain institutional roles and functions of branches of the federal government, states, citizens, media, parties, and other political actors 
\item Discuss the roles and power of citizens in American government and policymaking 
\item Apply knowledge of United States political history and political science theories to understand contemporary political events 
\item Evaluate activities of American political institutions and citizens, including their {\em de jure} powers and {\em de facto} operations 
\end{enumerate}

\section{Exam}
Students will be evaluated via an oral examination with a written synopsis based upon issues raised in the course. In preparation for the exam, students are expected to participate in weekly group presentations. These presentations will cover the week's reading material and involve leading a discussion on that material.

\section{Reading Material}
The assigned material for the course consists of empirical research articles and book chapters, all of which are available online or in the printed compendium. There is no textbook.

%\clearpage
\section{Schedule}
The general schedule for the course is as follows. Details on the readings for each week are provided on the following pages.

\secttoc

% FOREIGN POLICY?
% Graham Allison, “Conceptual Models and the Cuban Missile Crisis,” American Political Science Review, 1969. 63(3): 689-718.
% GroupThink
% Analogical reasoning
% Lawrence Jacobs and Benjamin Page, “Who Influences U.S. Foreign Policy?” American Political Science Review, 2005. 99 (1): 107-123.

\clearpage


\subsection{No class (Week 36)}
\vspace{1em}

\subsubsection*{Readings}
\reading{Noel2010} % The Forum
\reading[Ch.2 (62--98) from ]{DelliCarpiniKeeter1996}






\clearpage
\subsection{American Values and Opinions (Week 37)}
\emph{Topic}
\vspace{1em}

\subsubsection*{Readings}
\reading[Selections from ]{Tocqueville2000}
\reading{Hartz1991} % Hartz in Classic Readings volume
\reading{Smith1993} % multiple traditions
\reading{MacKuenEriksonStimson1989}
\reading{ConoverFeldman1981}
\reading{SullivanPiersonMarcus1979}

\seealso
\reading{Feldman1988}
\reading{PageShapiro1992}
% Manifest Destiny



\clearpage
\subsection{The American Founding (Week 38)}
\emph{What does the American Constitution of 1787 say? How does it distribute rights and powers among the branches of national government, states, and citizens? What controversies did the constitution create and resolve? How have those challenges been subsequently addressed?}

\vspace{1em}
\subsubsection*{Readings}
{\em On early colonization}:\\
-- Winthrop, John. 1630. A Model of Christian Charity.

{\em On independence from Britain}:\\
-- The Declaration of Independence (1776)\\
-- Paine, Thomas. (1776). Common Sense: Thoughts on the Present State of American Affairs. Available at: \url{http://www.ushistory.org/paine/commonsense/sense4.htm}
-- The Articles of Confederation (1778)\\

{\em On the American Constitution}:\\
-- The Constitution of the United States of America (1787)\\
\reading[Ch.1 (27--49) from ]{Maier2010}
\reading[1--2, 10, 45, 65, 69, 70 from ]{FederalistPapers}
\reading{Brutus} % Letters of Brutus

\seealso
\reading{Riker1988}
\reading{Riker1996}
\reading{Dahl2006}
\reading{Skowronek1982}
\reading{Skocpol1995}



\clearpage
\subsection{Congress (Week 39)}
\emph{What are the formal and informal institutions of Congress? How are Congressional decisions made?}
\vspace{1em}
\subsubsection*{Readings}
\reading{SchicklerWawro2011} % filibuster
\reading{Polsby1968}
\reading{ShepsleWeingast1994}

% lobbying
\reading{DenzauMunger1986}
\reading{Kollman1997}
\reading{Austen-SmithWright1994} % Counteractive Lobbying
\reading{HallWayman1990} % Moneyed interests
\reading{Stratmann2002} % Stratmann, Thomas. 2002. “Can Special Interests Buy Congressional Votes? Evidence from Financial Services Legislation.” Journal of Law and Economics 45: 345-373.

\reading{EdwardsBarrettPeake1997} % George C. Edwards III, Andrew Barrett, and Jeffrey Peake (1997), "The Legislative Impact of Divided Government," American Journal of Political Science

% Party influence
\reading{SnyderGroseclose2000} % Snyder, James M., and Timothy Groseclose. 2000. “Party Influence and Congressional Roll-Call Voting.” American Journal of Political Science 44: 193-211.
\reading{Krehbiel2000} % Krehbiel, Keith. 2000. “Party Discipline and Measures of Partisanship.” American Journal of Political Science 44: 177-192.
\reading{Krehbiel1993} % Krehbiel, Keith. 1993. Where’s the Party? British Journal of Political Science, 23: 235-66.


\reading[Ch.6 (129--159) from ]{HibbingTheissMorse2002}


\seealso
\reading{CoxMcCubbins1993}
\reading{Krehbiel1998}
\reading{Mayhew1974}
\reading{Fenno1978}
\reading{DiermeierVlaicu2011}


\clearpage
\subsection{The Presidency and Executive Branch (Week 40)}
\emph{What powers does the President have? And can the President do with those powers? How does the President respond to and influence the public?}
\vspace{1em}
\subsubsection*{Readings}
\reading{RagsdaleTheis1997}
\reading[Selections from ]{Neustadt1960}
\reading[Selections from ]{Cameron2000}
% John B. Gilmour (2011), \Political Theater or Bargaining Failure: Why Presidents Veto," Presidential Studies Quarterly

\reading{CanesWrone2001}
\reading{BaumKernell1999}
% Cohen, Jeffrey E. 1995. “Presidential Rhetoric and the Public Agenda.” American Journal of Political Science 39:87-107.


\reading{McCubbinsSchwartz1984}
\reading{Moe1987}

% Terry Moe, “Political Control and the Power of the Agent,” Journal of Law, Economics and Politics, 2005. 21(1): 1-29.

% Something on the imperial presidency; unitary executive
\reading{Cooper2005}
% Frontline: Cheney's War

\seealso
\reading{WhittingtonCarpenter2003}




\clearpage
\subsection{Courts and Judicial Decision Making (Week 41)}
\emph{What role do courts, and especially the Supreme Court, have in the American political process? How do courts influence policy?}

\vspace{1em}
\subsubsection*{Readings}
\reading{HowardSegal2002}

% Maybe JohnsonMartin1998
% Maybe Vince Hutchings paper

% Lee Epstein, Rene Lindstadt, Jerey A. Segal, and Chad Westerland (2006), "The Changing Dynamics of Senate Voting on Supreme Court Nominees," Journal of Politics

% Jeffrey A. Segal,“Separation of powers games in the positive theory of law and the courts,” American Political Science Review, vol. 91 (1997), pp. 28–44.
% Jeffrey A. Segal, and Harold J. Spaeth,“The influence of stare decisis on the votes of United States Supreme Court justices,” American Journal of Political Science, vol. 40 (1996), pp. 971-1003.
% Jack Knight and Lee Epstein, “The norm of stare decisis,” American Journal of Political Science, vol. 40 (1996), pp. 1018–1035.

% Ken I. Kersch, “The Reconstruction of Constitutional Privacy Rights and the New American State,” Studies in American Political Development 16 (Spring 2002), 61-87.
% Justin Crowe, “The Forging of Judicial Autonomy,” Journal of Politics 69 (February 2007): 73-87.

\reading{HuberGordon2004} % elections and judicial punitiveness

\reading{BinderMaltzman2002} % Senate delays in judicial appointments

\reading{GibsonCaldeiraSpence2003} % Bush v. Gore and legitimacy

\seealso
\reading{Dahl1957} % Journal of Public Law
% McCulloch v. Maryland







\clearpage
\subsection{No class (Week 42)}

\clearpage
\subsection{State and Local Governments (Week 43)}
\emph{Topic}

\vspace{1em}
\subsubsection*{Readings}

% Something about turnout

% Gerald Wright and Brian Schaffner, “The Influence of Party: Evidence from the State Legislatures,” American Political Science Review, 2002. 96 (2): 367-379.
\reading{Volden2006} % Craig Volden, “States as Policy Laboratories: Emulating Success in the Children’s Health Insurance Program,” American Journal of Political Science, 2006. 50 (2): 294-312.

\reading{GerberHopkins2011} % Gerber, Elisabeth R., and Daniel Hopkins. 2011. “When Mayors Matter: Estimating the Impact of Mayoral Partisanship on City Policy.” American Journal of Political Science 55: 326-339. 


\reading{Rigby2012} % Obamacare resistance
\reading{LaxPhillips2012} % The Democratic Deficit in the States
% Something on local politics
% Dahl New Haven

% something by Eric Oliver

\seealso





\clearpage
\subsection{Partisan Politics (Week 44)}
\emph{Topic}
\vspace{1em}

\subsubsection*{Readings}
% Lipset: Why no socialism in America?; Eric Foner 1984
\reading{APSA1950} % APSA Report
\reading{Fiorina1980} % Daedalus
% Aldrich
% Bawn et al. Theory of Parties
\reading{KogerMasketNoel2010}
\reading{IyengarSoodLelkes2012} % Affective Polarization
\reading{MacKuenEriksonStimson1989}

% something about origins of party identificaiton
% something about vote choice ( American Voter? )

\seealso
\reading{GreenPalmquistSchickler2004}








\clearpage
\subsection{Campaigning and Elections (Week 45)}
\emph{Topic}
\vspace{1em}
\subsubsection*{Readings}
\reading{McDonaldPopkin2001}
\reading{Burdenetal2014}
\reading{LauRovner2009}

\reading{Tufte1975} % midterms

% mobilization
\reading{EnosFowlerVavreck2014}

\reading{Gelmanetal2007}
% electoral college
\reading{HuddyTerkildsen1993} % Gender stereotypes

% Citizens United v. Federal Election Commission (2010)

\reading{McDonald2004} % Comparative Analysis of Redistricting Institutions



% 1950 Party Platforms and 2012 party platforms

\seealso
\reading{Key1955} % or Burnham
\reading{FridkinKenneyWoodall2008}
\reading{Brooks2013} % He Runs, She Runs
\reading{Geer2006} % In Defense of Negativity
\reading{AnsolabehereIyengar1997}
\reading{Gerberetal2011}
\reading{GerberGreen2000}
\reading{Erikson1978} % Erikson, Robert S. 1978. “Constituency Opinion and Congressional Behavior: A Reexamination of the Miller-Stokes Representation Data.” American Journal of Political Science 22(3): 511-535.
\reading{Hillygus2005}






\clearpage
\subsection{Elite and Mass Polarization (Week 46)}
\emph{Topic}
\vspace{1em}
\subsubsection*{Readings}

\reading{APSA1950} % APSA Report
\reading{FiorinaAbrams2008}
\reading{Levendusky2009}
\reading{PooleRosenthal1984}
% primaries
\reading{Hetherington2001}
\reading{DruckmanPetersonSlothuus2013}

\reading{McCartyPooleRosenthall2009} % McCarty, N., Poole, K. T. and Rosenthal, H. (2009), Does Gerrymandering Cause Polarization?. American Journal of Political Science, 53: 666–680. doi: 10.1111/j.1540-5907.2009.00393.x

% network homophily
\reading{Mutz2002}

\seealso
\reading{Bishop2008} % The Big Sort
\reading{Abramowitz2008} % primaries
\reading{MutzReeves2005}
\reading{ArceneauxJohnsonMurphy2012}






\clearpage
\subsection{Inequality (Week 47)}
\emph{Topic}
\vspace{1em}
\subsubsection*{Readings}
-- APSA Task Force on Inequality and American Democracy. (2006). American Democracy in an Age of Rising Inequality. Washington, DC: American Political Science Association. Available from: \url{http://www.apsanet.org/imgtest/taskforcereport.pdf}
\reading[Ch ??? from ]{Bartels2008}
% Domhoff
\reading{PageBartelsSeawrigth2013}
\reading{PageShapiro1983}

\reading{VerbaBurnsSchlozman2003} % Unequal at the Starting Line

% Something about power of business
% Citizens' United

% Something about the welfare state?

% Milton Friedman, Capitalism and Freedom

\seealso
\reading{WintersPage2009}








\clearpage
\subsection{Politics of Race and Ethnicity (Week 48)}
\emph{Topic}
\vspace{1em}
\subsubsection*{Readings}
\readings{HutchingsValentino2004}
\readings[Volume I, Chapter 10 (316--407) from ]{deTocqueville2000}
\reading{Gilens1996} % “‘Race Coding’ and White Opposition to Welfare.”
\reading{Kuklinskietal1997} % “Racial Prejudice and Attitudes toward Affirmative Action,”
\reading{SearsHenslerSpeer1979} % Whie Opposition to Busing
\reading{Mansbridge1999} % “Should Blacks Represent Blacks and Women Represent Women? A Contingent ‘Yes,’”
\reading{CameronEpsteinOHalloran1996} % “Do Majority-Minority Districts Maximize Black Substantive Representation in Congress?”
\reading{WilliamsCollins2001} % racial disparities in health
\reading{PettitWestern2004} % incarceration

% Brown v. Board of Education (1954)


\seealso
% Takaki, Different Mirror
\reading{KatznelsonMettler2008}
\reading{Mendelberg2010}
\reading{KinderSanders1996}
\reading{ButlerBroockman2011}
\reading{ValentinoBraderJardina2013}
\reading{CohenDawson1993}
% Fitzhugh: Cannibals All?
% McAdams on black insurgency
% Chong on civil rights movement
% Gilens - Why Americans Hate Welfare
% Kinder vs. Sniderman








\clearpage
\subsection{Policy Controversies I (Week 49)}
\emph{Topic}
\vspace{1em}
\subsubsection*{Readings}
\reading{Hacker1998} % health policy
\reading{MulliganGrantBennett2012}
\reading{Adams1997}
\reading{Bartels2005} % Homer gets a tax cut
% District of Columbia v. Heller (2008)
% Lawrence v. Texas (2003)

\reading{Kuklinskietal2000} % welfare information
\reading{NyhanReifler2010} % misinfo
\readings{Gainesetal2007} % partisan interpretation of Iraq


\seealso
\reading{LaxPhilips2009}
\reading{NaginPepper2012} % NAP Report: DETERRENCE AND THE DEATH PENALTY

% United States vs. Windsor (2013) and Scalia’s Dissent

% Terrorism: Hamdi v. Rumsfeld/Hamdan v. Rumsfeld

% Torture Memos/John Yoo
% Addington/Executive privilege




\clearpage
\subsection{Policy Controversies II (Week 50)}
\emph{Topic}
\vspace{1em}
\subsubsection*{Readings}


\seealso




\clearpage
\subsection{Wrap-up (Week 51)}
\emph{Topic}
% Have students review
\vspace{1em}
\subsubsection*{Readings}




% load bibtex, but don't generate a bibliography
\bibliographystyle{plain}
\nobibliography{Syllabi}

\end{document}
