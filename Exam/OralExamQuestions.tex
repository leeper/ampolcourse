\documentclass[12pt,a4paper]{article}
\usepackage[top=1in, bottom=1in, left=1in, right=1in]{geometry}
\usepackage{setspace,hyperref,mdwlist}
\setlength\parindent{0pt}
\usepackage[T1]{fontenc}
\usepackage{lmodern}
\hypersetup{
    bookmarks=true,         % show bookmarks bar?
    unicode=false,          % non-Latin characters in Acrobat’s bookmarks
    pdftoolbar=true,        % show Acrobat’s toolbar?
    pdfmenubar=true,        % show Acrobat’s menu?
    pdffitwindow=false,     % window fit to page when opened
    pdfstartview={FitH},    % fits the width of the page to the window
    pdftitle={Exam: Issues in American Politics and Government},    % title
    pdfauthor={Thomas J. Leeper},     % author
    pdfsubject={Political Science},   % subject of the document
    pdfnewwindow=true,      % links in new window
    pdfborder={0 0 0}
}

\title{Oral Exam Questions}
\author{}
\date{}

\begin{document}

\maketitle

\onehalfspacing

\begin{enumerate}
\item How does a bill become law in the United States and why is that process so difficult?
	\begin{itemize*}
	\item President
	\item Congress
		\begin{itemize*}
		\item Bicameralism
		\item Conference committees
		\item Lobbying
		\item Public opinion
		\item Ideology
		\item Leadership
		\item Chamber rules (Senate and House)
		\end{itemize*}
	\item Supreme Court
	\item States
	\end{itemize*}
\item To what extent is the United States an equal society? What consequences does this level of equality have?
	\begin{itemize*}
	\item Economic inequality
	\item Political inequality
	\item Racial inequality
	\item Welfare state policies
	\end{itemize*}
\item Would you say the American federal government does enough to reduce inequalities?
	\begin{itemize*}
	\item Facts of inequality?
	\item Public opinion on inequality
	\item Public opinion on taxes
	\end{itemize*}
\item To what extent is the United States a politically polarized society? What consequences does this have?
	\begin{itemize*}
	\item What does polarization mean? What about sorting?
	\item Is it polarized at the elite or mass levels?
	\item Media
	\item Social relationships (marriage, friendships, etc.)
	\item Elections, extremity, redistricting
	\end{itemize*}
\item What role do states play in the development of policy in the United States? And how do states and the federal government interact in the policymaking process?
	\begin{itemize*}
	\item State-by-state learning
	\item National government adoption of state policies
	\item Federalism; separation of powers
	\item Supreme Court involvement in states
	\end{itemize*}
\item How much and what kinds of impact do parties have on politics and policymaking in the United States?
	\begin{itemize*}
	\item Congress: party discipline, leadership, floor control, divided government
	\item Presidency: president as party leader
	\item Parties as electoral competitors
	\item Partisanship in the electorate
	\end{itemize*}
\item What are the distinguishing features of American campaigns and elections? What impact do those features have on turnout and vote choice?
	\begin{itemize*}
	\item Plurality rule
	\item Two-party system
	\item Large number of races
	\item Non-competitive races and uncontested races
	\item Nonpartisan elections
	\item Judicial elections
	\item Direct democracy
	\end{itemize*}
\item Why aren't there third parties in the United States?
	\begin{itemize*}
	\item Plurality rule elections
	\item Polarization
	\end{itemize*}
\item The Supreme Court is an important institution in American politics. What role does it play in politics?
	\begin{itemize*}
	\item Judicial review
	\item What are the originals of judicial review
	\item How does it influence policymaking? (e.g., strategic interactions)
	\end{itemize*}
\item How does the Supreme Court decide cases? What factors influence their decisions?
	\begin{itemize*}
	\item Ideology
	\item Precedent
	\item Arguments
	\item Strategic considerations
	\item Public legitimacy
	\end{itemize*}
\item There are lots of different ways of defining liberty. Friedman, for example, has one definition of liberty. Tocqueville talks a lot about liberty. The American Founders also talked about liberty. Would you say the United States a liberal society?
	\begin{itemize*}
	\item Declaration of Independence and founding principles
	\item Tocqueville on equality
	\item Evidence on equality and inequality
	\item Friedman's notion of liberalism
	\item Liberalism as a political ideology in the United States
	\end{itemize*}
\item What do you think you can learn from American politics that might be relevant for your home country?
\end{enumerate}

\end{document}