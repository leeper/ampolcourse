\documentclass[12pt,a4paper]{article}
\usepackage[top=1in, bottom=1in, left=1in, right=1in]{geometry}
\usepackage{setspace,hyperref}
\usepackage[T1]{fontenc}
\usepackage{lmodern}
\hypersetup{
    bookmarks=true,         % show bookmarks bar?
    unicode=false,          % non-Latin characters in Acrobat’s bookmarks
    pdftoolbar=true,        % show Acrobat’s toolbar?
    pdfmenubar=true,        % show Acrobat’s menu?
    pdffitwindow=false,     % window fit to page when opened
    pdfstartview={FitH},    % fits the width of the page to the window
    pdftitle={Exam: Issues in American Politics and Government},    % title
    pdfauthor={Thomas J. Leeper},     % author
    pdfsubject={Political Science},   % subject of the document
    pdfnewwindow=true,      % links in new window
    pdfborder={0 0 0}
}

\title{Final Exam: Issues in American Politics and Government}
\author{Thomas J. Leeper\\
Department of Political Science and Government\\
Aarhus University}

\begin{document}

\maketitle

The exam for this course consists of an oral exam with a written synopsis. The oral examination takes about 25 minutes. The synopsis should be about 800-1200 words (2-3 type-written pages). During the oral examination, about one half of the time will be spent on the synopsis and the other half will be spent on topics covered in the course as a whole. The exam will take place in English and the synopsis must be written in English.

For the written synopsis, each student should analyze a contemporary political debate with the goal of explaining \dots...

[something here]

Factors to consider in your analysis might include:

\begin{itemize}
\item Political institutions and the relationships between institutions
\item Particular politicians and political actors involved in the debate
\item Public opinion and American political values or culture
\item Issues of constitutionality
\item Issue framing and arguments on each side of the debate
\end{itemize}


\end{document}