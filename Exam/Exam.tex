\documentclass[12pt,a4paper]{article}
\usepackage[top=1in, bottom=1in, left=1in, right=1in]{geometry}
\usepackage{setspace,hyperref,mdwlist}
\setlength\parindent{0pt}
\usepackage[T1]{fontenc}
\usepackage{lmodern}
\hypersetup{
    bookmarks=true,         % show bookmarks bar?
    unicode=false,          % non-Latin characters in Acrobat’s bookmarks
    pdftoolbar=true,        % show Acrobat’s toolbar?
    pdfmenubar=true,        % show Acrobat’s menu?
    pdffitwindow=false,     % window fit to page when opened
    pdfstartview={FitH},    % fits the width of the page to the window
    pdftitle={Exam: Issues in American Politics and Government},    % title
    pdfauthor={Thomas J. Leeper},     % author
    pdfsubject={Political Science},   % subject of the document
    pdfnewwindow=true,      % links in new window
    pdfborder={0 0 0}
}

\title{Final Exam:\\Issues in American Politics and Government}
\author{}
\date{}

\begin{document}

\maketitle

The exam for this course consists of an oral exam with a written synopsis. The oral examination takes about 25--30 minutes. The synopsis should be about 800-1200 words (2-3 type-written pages). During the oral examination, about one half of the time will be spent on the synopsis and the other half will be spent on topics covered in the course as a whole. The exam will take place in English and the synopsis must be written in English.\\

For the written synopsis, each student should select a contemporary political issue, debate, controversy, or policy question in the United States, briefly describe the issue and why it is controversial in the United States, and then assess the role of different institutional, legal, cultural, historical, and/or attitudinal factors involved in the controversy. Factors to consider in your analysis might include:

\begin{itemize*}
\item Political institutions and the relationships between institutions
\item Federalism and the division of responsibilities between local, state, and national governments
\item The role of particular politicians and political actors (e.g., activists, parties, interest groups, etc.)
\item Public opinion, partisanship, ideology, and American political values or culture
\item Issues of constitutionality, legality, and judicial review
\item Framing and arguments on each side of the controversy
\item Inequalities of race, gender, income, class, etc.
\item American history and the contents of extant legislation and prior political debates
\end{itemize*}

The synopsis does not need to address all of these facets of the debate. Instead, the document should highlight the factors most important to: the existence of the controversy, political debate surrounding the controversy, and any relevant policy activity.\\

Students will have the opportunity to present, defend, and receive feedback on the written synopsis during the final class sessions.\\

\end{document}