\documentclass[a4, 12pt]{article}
\usepackage[top=2cm, bottom=2cm, left=2cm, right=2cm]{geometry}
\usepackage{setspace}

\title{State-to-State Policy Variations\vspace{-2em}}
\author{}
\date{}

\begin{document}

\maketitle

\onehalfspacing

\noindent One of the peculiar implications of American federalism is substantial state-to-state variation in policy. Because Congress is, in large part, limited in policymaking authority to the enumerated powers granted by the Constitution (esp. Article I, Section 8) and because the 10th Amendment reserves all non-enumerated powers to the states, states have sole or primary authority over a large number of policy domains. The result is that states are often seen as ``laboratories of democracy'' that experiment with alternative policies and learn from other states.

\vspace{1em}
\noindent Purpose: The purpose of this activity is to examine some of these state-to-state policy variations in order to understand the impact of federalism on the day-to-day lives of individuals living in the United States. Furthermore, the exercise gives you an opportunity to start thinking about the difficulty of succinctly describing the state of policy in the United States.

\vspace{1em}
\noindent Your task: Working with a partner, select one of the policy areas listed below (or another policy area, if there is one you find more interesting). You should reach a consensus with your classmates about which pair will work on which topic, so that only one group covers each topic. Research the policy area you have chosen and determine the set of policy alternatives implemented in the various states. How much variation is there? What different kinds of policies exist in this domain? What features are common across states and what features are different? Then, try to make sense of these variations. Possibly consider the following: Why are there variations across states? When did those variations emerge? Are policy differences the result of state-level lawmaking (i.e., legislation enacted by state legislatures), the result of judicial decisions (i.e., state or federal court decisions about the legality and constitutionality of particular policies), or some combination of both? Are policies in this domain changing or are they relatively stable? What are the major arguments about the policy and who is active in pressuring for policy changes?

\vspace{1em}
\noindent Prepare a short (about 5 minutes) presentation for next week's class about these state-to-state variations, highlighting what you see as the most interesting and most important aspects of the policy domain.

\clearpage
\section*{Possible Topics}

\begin{itemize}\itemsep1em
\item Gun laws (e.g., ``conceal carry'' vs. ``open carry'' status; age restrictions)
\item Death penalty laws and use
\item Cigarette and alcohol laws (e.g., taxation, minimum age of purchase, drunk-driving laws) 
\item Rights of gays and lesbians (e.g., marriage, job protections, right to adopt children)
\item Legal status of abortion
\item Term limits for elected officials
\item Tax rates (e.g., on income, property, sales/VAT, gasoline)
\item Legal status of marijuana
\item The minimum hourly wage for employees
\item Laws governing judiciary (e.g., are judges elected or appointed? are partisan endorsements allowed?)
\item Laws regulating voting (e.g., voter registration requirements, ``Voter ID'' laws, types of ballots or voting technology used, etc.)
\item Trade union ``right to work'' laws and status of ``open'' and ``closed'' shop rules
\item Primary and secondary education standards (see, e.g., ``No Child Left Behind'' and ``Common Core'')
\end{itemize}

\vspace{1em}
\noindent Note: You can also pick another policy domain, if there is one you find more interesting. Again, make sure only one group works on any particular topic.


\end{document}
