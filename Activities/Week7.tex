\documentclass[a4, 12pt]{article}
\usepackage[top=2cm, bottom=2cm, left=2cm, right=2cm]{geometry}
\usepackage{setspace}

\title{Electoral Politics in the U.S.\vspace{-2em}}
\author{}
\date{}

\begin{document}

\maketitle

\onehalfspacing

\noindent Elections in the United States are incredibly diverse in terms of the issues debated, the level of public engagement, and the amount of money spent on electoral campaigns. Common themes, by contrast, are two-party competition, negative tone, and a high use of campaign advertisements on television, radio, the internet, and in other media.

\vspace{1em}
\noindent Purpose: The purpose of this activity is to understand how political campaigns and elections work in the United States. You should be able to identify the most important features of American political campaigns, analyze the importance of money, advertising, and party competition on election outcomes, and critique the value of campaigns as a tool of democratic control.

\vspace{1em}
\noindent Your task: You have been assigned an election for a U.S. Senate seat, a U.S. House of Representatives seat, or a state Governorship. Be prepared to discuss the campaigns with your classmates in class next week. Toward that end, please investigate your assigned campaign and attempt to answer the following questions:

\begin{itemize}
	\item What office was the election for?
	\item Who were the candidates? Which parties are they from? Who was/were the incumbents?
	\item Who won? What were the election results? What was turnout?
	\item What did the primary election look like? How close were the primaries?
	\item What were the issues debated in the campaign?
	\item What was the campaign like? Was it positive or negative in tone?
	\item How much money was spent on the campaign? Who was spending that money?
\end{itemize}

\noindent Use resources available to you online and the research articles assigned for next week's class to understand what happened in the election campaign. You will want to use the candidates' campaign websites, local or national news media, and political commentary and data websites (e.g., FiveThirtyEight, New York Times Upshot, HuffingtonPost Pollster, Real Clear Politics, Politico, etc.). I am particularly interested in your reactions to the following:

\begin{itemize}
	\item What was surprising about the campaign?
	\item What was confusing about the campaign?
	\item What made the campaign interesting or important?
\end{itemize}



\end{document}
