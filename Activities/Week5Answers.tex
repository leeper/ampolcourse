\documentclass[a4, 12pt]{article}
\usepackage[top=2cm, bottom=2cm, left=2cm, right=2cm]{geometry}
\usepackage{mdwlist}

%opening
\title{(Some) Important Supreme Court Decisions Answers\vspace{-2em}}
\author{}
\date{}

\begin{document}

\maketitle

\begin{enumerate}\itemsep1em
\item \textit{Marbury v. Madison} (1803)
\begin{itemize*}
	\item Marbury seeks deserved, but denied commission
	\item Marbury challenges government at SCOTUS
	\item Court sides with Marbury on substance, overruling Madison
	\item But, Court rules Congress didn't have authority to grant SCOTUS original jurisdiction in this case
	\item Consequence: established judicial review
\end{itemize*}

\item \textit{McCulloch v. Maryland} (1819) % necessary and proper
\begin{itemize*}
		\item Maryland tries to tax the Second Bank of the United States
		\item McCulloch (head of bank) refuses to pay tax
		\item Maryland asserts bank is unconstitutional because Congress is not given explicit power to create banks
		\item Court overrules due to the ``necessary and proper'' clause (Article 1, Sec. 8, clause 18)
\end{itemize*}

\item \textit{Gibbons v. Ogden} (1824) % interstate commerce
\begin{itemize*}
	\item Some steamboat operators fight over shipping routes, with one trying to prevent another from using a boat on a monopolized route
	\item State court issues injunction because state law regulates shipping
	\item SCOTUS overrules and says only Congress has the right to regulate interstate commerce
\end{itemize*}

\item \textit{Dred Scott v. Sandford} (1857) % slaves are property
\begin{itemize*}
	\item Dred Scott is a slave taken into free territory by owner
	\item SCOTUS denied the claim because Scott did not have standing because, as a slave, he was not a citizen
	\item Furthermore, Missouri Compromise is unconstitutional
\end{itemize*}

\item \textit{Plessy v. Ferguson} (1896) % separate but equal
\begin{itemize*}
	\item Louisiana citizens challenge racially segregated rail cars
	\item SCOTUS rules that 14th amendment requires equal treatment, but ``separate but equal'' facilities satisfy this
\end{itemize*}

\item \textit{Brown v. Board of Education} (1954) % integrated schools
\begin{itemize*}
	\item Class action suit brought against Topeka, Kansas schools, challenging state law that permitted (but not required) separate school facilities based on race
	\item SCOTUS combined five cases
	\item Unanimous decision: segregation was unconstitutional
\end{itemize*}

\item \textit{Schenck v. United States} (1919) % clear and present danger
\begin{itemize*}
	\item Socialist party members distribute flyers protesting conscription
	\item Convicted under the Espionage Act for dissent
	\item SCOTUS upholds conviction that 1st amendment does not protect speech that presents ``a clear and present danger'' (i.e., encourages the commission of a crime)
\end{itemize*}

\item \textit{Miller v. California} (1973) % obsenity
\begin{itemize*}
	\item Miller produces pornography is convicted of obsenity
	\item SCOTUS rules obsenity is not protected by 1st amendment
	\item But! establishes rules for what is considered obsenity: ``lacks serious literary, artistic, political, or scientific value''
\end{itemize*}

\item \textit{Engel v. Vitale} (1962) % school prayer is establishment
\begin{itemize*}
	\item Jewish families protest voluntary school-sponsored prayer
	\item SCOTUS rules school-sponsored prayers violated the Establishment clause
\end{itemize*}

\item \textit{Tinker v. Des Moines Independent Community School District} (1969) % armbands are protected free speech
\begin{itemize*}
	\item Students protest the Vietnam war by wearing black armbands
	\item Des Moines schools ban armbands
	\item SCOTUS rules 1st amendment applies to schools
\end{itemize*}

\item \textit{Gideon v. Wainwright} (1963) % right to counsel
\begin{itemize*}
	\item Gideon is charged with a non-capital crime but cannot afford an attorney
	\item SCOTUS rules that right to counsel is protected by fifth and sixth amendments
\end{itemize*}

\item \textit{Miranda v. Arizona} (1966) % rights at time of arrest
\begin{itemize*}
	\item Miranda is arrested on charges of rape and confesses, but is not advised of his legal rights (e.g., counsel, self-incrimination)
	\item SCOTUS rules confession is inadmissible because suspect must be aware of his rights
	\item ``Miranda warning'': rights read to anyone arrested in the United States
\end{itemize*}

\item \textit{Furman v. Georgia} (1972) % death penalty unconstitutional for non-homicide cases
\begin{itemize*}
	\item Consolidation of two cases involving the death penalty
	\item SCOTUS rules death penalty is cruel and unusual punishment in these cases, but provides no majority rationale
	\item De facto death penalty moratorium until \textit{Gregg v. Georgia} (1976)
\end{itemize*}

\item \textit{Roe v. Wade} (1973) % abortion
\begin{itemize*}
	\item Abortion is illegal and Roe attempts to have an abortion of a pregnancy conceived via rape
	\item SCOTUS rules that abortion is protected by 14th amendment right to due process, specifically a non-enumerated ``right to privacy''
	\item Most significant case from the perspective of contemporary political controversy
\end{itemize*}

\item \textit{Burwell v. Hobby Lobby Stores, Inc.} (2014) % religious rights of corporations
\begin{itemize*}
	\item ACA/Obamacare requires corporations to provide health insurance that includes contraceptive coverage
	\item Hobby Lobby challenges that it violates their 1st amendment rights to religious freedom
	\item SCOTUS agrees with Hobby Lobby essentially on technical grounds: government can restrict religious liberty but only using the ``least restrictive'' method to implement the government's interest
\end{itemize*}

\item \textit{United States v. Windsor} (2013) % gay marriage
\begin{itemize*}
	\item Windsor is married to a another woman, who dies leaving her estate to Windsor and wants a tax exemption only available to married couples
	\item She is not considered married under federal Defense of Marriage Act (DOMA)
	\item SCOTUS rules that DOMA is unconstitutional because it violates due process (5th Amendment)
	\item Decision same day as a ruling that allowed a California state constitutional amendment to be overruled by state courts
\end{itemize*}

\item \textit{District of Columbia v. Heller} (2008) % 2nd Amendment applies to individuals
\begin{itemize*}
	\item DC has a universal ban on guns and Heller challenges
	\item SCOTUS rules that 2nd Amendment applies to gun ownership unconnected to the militia
	\item Expanded later by \textit{McDonald v. Chicago} (2010)
\end{itemize*}

\item \textit{Hamdi v. Rumsfeld} (2004) % Habeas protections
\begin{itemize*}
	\item Hamdi is a U.S. citizen held in Guantanamo Bay as an enemy combatant
	\item SCOTUS rules that Hamdi, as a U.S. citizen, has a right to due process protection; the legality of enemy combatant status being unresolved
	\item Hamdi eventually deported and forced to give up his citizenship
\end{itemize*}

\item \textit{Hamdan v. Rumsfeld} (2006) % Guantanamo violates Geneva convention
\begin{itemize*}
	\item Hamdan drives a car for Osama bin Laden and is held as an enemy combatant in Guantanamo Bay
	\item SCOTUS rules (5-3) that Bush administration did not have authority from Congress to establish military tribunals because they failed to comply with the Geneva convention
	\item \textit{Boumediene v. Bush} later rules military tribunals deny habeas, so detainees can challenge their detention in civilian courts
\end{itemize*}

\item \textit{United States v. Nixon} (1974) % executive privilege exists but does not apply to criminal cases
\begin{itemize*}
	\item Nixon refused to release tapes related to Watergate
	\item SCOTUS rules that ``executive privilege'' does not protect evidence of criminal wrongdoing
	\item Nixon resigns
\end{itemize*}

\item \textit{Clinton v. City of New York} (1998) % line item veto is unconstitutional
\begin{itemize*}
	\item Republican's Line Item Veto Act of 1996 grants line-item veto to the President
	\item Clinton uses line-item veto on a budget bill
	\item SCOTUS rules that constitution does not give President independent authority to author or repeal legislation, so line-item veto is unconstitutional; Congress cannot change its own authority or the authority of other branches
\end{itemize*}

\item \textit{Sony Corp. of America v. Universal City Studios, Inc.} (1984) % VCRs legal
\begin{itemize*}
	\item Sony invents the betamax VCR and is sued by the entertainment industry for facilitating copyright infringement
	\item SCOTUS rules: (1) VCR has non-infringement uses and (2) the use of VCRs constitutes ``fair use''
\end{itemize*}

\item \textit{National Federation of Independent Business v. Sebelius} (2012) % ACA/Obamacare constitutional, but medicaid expansion is not
\begin{itemize*}
	\item ACA/Obamacare challenged on multiple grounds: (1) Medicaid expansion coercive, (2) individual insurance mandate is unconstitutional
	\item Administration argues that mandate is constitutional under commerce clause
	\item SCOTUS rules Medicaid expansion rules are coercive; but individual mandate is valid under the taxing power not commerce clause
\end{itemize*}

\item \textit{Citizens United v. Federal Election Commission} (2010) % federal campaign donation limits are unconstitutional
\begin{itemize*}
	\item Conservative group airs film critical of Hillary Clinton; McCain--Feingold prohibits prohibits such things, especially by corporations and unions
	\item SCOTUS rules that McCain--Feingold is an unconstitutional prohibition on free speech by corporations and unions
	\item Much more corporate spending on third-party advertising, but donations directly to campaigns and candidates remains illegal
\end{itemize*}

\item \textit{Bush v. Gore} (2000) % recount unconstitutional
\begin{itemize*}
	\item Bush wins Florida, but Gore challenges for a recount
	\item Recounts are slow and Florida certifies results before recounts finish
	\item Various challenges
	\item SCOTUS rules that recounts violated equal protection clause, but no alternative recount could satisfy it and state election laws, so results stand
	\item Bush wins
\end{itemize*}

\end{enumerate}


\end{document}
