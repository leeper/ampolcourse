\documentclass[12pt,a4paper]{article}
\usepackage[top=1in, bottom=1in, left=1in, right=1in]{geometry}
\usepackage{graphicx,setspace,hyperref,amsmath,amsfonts,multirow,ccaption,mdwlist,comment}
% mini table of contents
\usepackage{minitoc}
\dosecttoc % make section toc
\setcounter{secttocdepth}{2} % subsection depth
\renewcommand{\stctitle}{} % no title
\nostcpagenumbers

% optionally include commented environments
\excludecomment{lessonplan}

\setlength{\marginparwidth}{.5in}
\usepackage{natbib}
% Two lines to create in-text full citations for a syllabus
% And comment out my other standard bibtex commands
\usepackage{bibentry}
\newcommand{\reading}[2][]{\noindent --{#1} from \bibentry{#2}.\vspace{.25em}\\}
\newcommand{\textbook}[2][]{\noindent --{#1} from Groves et al.\vspace{.25em}\\} % textbook reference
\newcommand{\seealso}{\noindent \emph{See Also:}\\}
\newcommand{\topic}[1]{\noindent \textbf{#1}\\}
\usepackage[T1]{fontenc}
\usepackage{lmodern}
\hypersetup{
    bookmarks=true,         % show bookmarks bar?
    unicode=false,          % non-Latin characters in Acrobat’s bookmarks
    pdftoolbar=true,        % show Acrobat’s toolbar?
    pdfmenubar=true,        % show Acrobat’s menu?
    pdffitwindow=false,     % window fit to page when opened
    pdfstartview={FitH},    % fits the width of the page to the window
    pdftitle={Syllabus: Issues in American Politics and Government},    % title
    pdfauthor={Thomas J. Leeper},     % author
    pdfsubject={Political Science},   % subject of the document
    pdfnewwindow=true,      % links in new window
    pdfborder={0 0 0}
}

\title{Issues in American Politics and Government }
\author{Thomas J. Leeper\\
Department of Political Science and Government\\
Aarhus University}

\begin{document}
\nobibliography*

\maketitle

\faketableofcontents

%\section{Introduction}

The United States is a unique political system. It is one of the longest-running democracies in the world, has a relatively rare presidential system, is populated by a broad mix of racial, ethnic, religious, economic, and cultural groups, and takes an aggressive, frequently independent, role in other countries' affairs. This seminar dives into several important aspects of American democracy and politics to understand what shapes political activity in the United States. Students will leave the course with a deep understanding of the institutional, historical, philosophical, and cultural factors that shape American politics and will be able to better analyze policymaking and political events in the United States as a result. Broadly the course asks students to consider why things are the way they are in the United States and why things happen the way they do. In addressing these questions, the emphasis is placed on answering the questions `who has power in the United States?' and `what do they do with it?'


\section{Objectives}
The learning objectives for the course are as follows. By the end of the course, students should be able to:

\begin{enumerate}
\item Identify and explain dominant themes that shape (and have shaped) the dynamics of American politics from the founding to the present 
\item Describe political polarization in the contemporary United States, as well as its origins and political effects 
\item Describe political and economic inequalities in the United States and their consequences for political activity and policymaking 
\item Explain institutional roles and functions of branches of the federal government, states, citizens, media, parties, and other political actors 
\item Discuss the roles and power of citizens in American government and policymaking 
\item Apply knowledge of United States political history and political science theories to understand contemporary political events 
\item Evaluate activities of American political institutions and citizens, including their {\em de jure} powers and {\em de facto} operations 
\end{enumerate}

\section{Exam}
Students will be evaluated via an oral examination with a written synopsis based upon issues raised in the course. In preparation for the exam, students are expected to participate in weekly group presentations. These presentations will cover the week's reading material and involve leading a discussion on that material.

\section{Reading Material}
The assigned material for the course consists of empirical research articles and book chapters, all of which are available online or in the printed compendium. There is no textbook.

%\clearpage
\section{Schedule}
The general schedule for the course is as follows. Details on the readings for each week are provided on the following pages.

\secttoc

% FOREIGN POLICY?

\clearpage


\subsection{No class (Week 36)}
\emph{Topic}
\vspace{1em}

\subsubsection*{Assignment Due}

\subsubsection*{Readings}

\seealso




\clearpage
\subsection{The American Founding (Week 37)}
\emph{Topic}

\vspace{1em}
\subsubsection*{Assignment Due}

\subsubsection*{Readings}
-- The Declaration of Independence (1776)\\
-- The Constitution of the United States of American (1787)\\
% Federalist papers
% Paine?

\seealso




\clearpage
\subsection{Congress (Week 38)}
\emph{Topic}
\vspace{1em}
\subsubsection*{Assignment Due}

\subsubsection*{Readings}

\seealso



\clearpage
\subsection{The Presidency and Executive Branch (Week 39)}
\emph{Topic}
\vspace{1em}
\subsubsection*{Assignment Due}

\subsubsection*{Readings}

\seealso

\subsubsection*{In-class Activities}



\clearpage
\subsection{Courts and Judicial Decision Making (Week 40)}
\emph{Topic}

\vspace{1em}
\subsubsection*{Assignment Due}

\subsubsection*{Readings}

\seealso



\clearpage
\subsection{State and Local Governments (Week 41)}
\emph{Topic}
\vspace{1em}
\subsubsection*{Assignment Due}

\subsubsection*{Readings}

\seealso


\clearpage
\subsection{No class (Week 42)}

\clearpage
\subsection{Political Parties (Week 43)}
\emph{Topic}
\vspace{1em}

\subsubsection*{Assignment Due}

\subsubsection*{Readings}

\seealso



\clearpage
\subsection{Campaigns and Elections (Week 44)}
\emph{Topic}
\vspace{1em}
\subsubsection*{Assignment Due}

\subsubsection*{Readings}

\seealso




\clearpage
\subsection{Political Participation (Week 45)}
\emph{Topic}
\vspace{1em}
\subsubsection*{Assignment Due}

\subsubsection*{Readings}


\seealso




\clearpage
\subsection{Economic Inequality (Week 46)}
\emph{Topic}
\vspace{1em}
\subsubsection*{Assignment Due}

\subsubsection*{Readings}

\seealso



\clearpage
\subsection{Elite and Mass Polarization (Week 47)}
\emph{Topic}
\vspace{1em}
\subsubsection*{Assignment Due}

\subsubsection*{Readings}

\seealso




\clearpage
\subsection{Politics of Race and Ethnicity (Week 48)}
\emph{Topic}
\vspace{1em}
\subsubsection*{Assignment Due}

\subsubsection*{Readings}

\seealso





\clearpage
\subsection{Values and Opinions (Week 49)}
\emph{Topic}
\vspace{1em}
\subsubsection*{Assignment Due}

\subsubsection*{Readings}


\clearpage
\subsection{Media Politics (Week 50)}
\emph{Topic}
\vspace{1em}
\subsubsection*{Assignment Due}

\subsubsection*{Readings}


\clearpage
\subsection{ (Week 51)}
\emph{Topic}
\vspace{1em}
\subsubsection*{Assignment Due}

\subsubsection*{Readings}



% load bibtext, but don't generate a bibliography
\bibliographystyle{plain}
\nobibliography{Syllabi}

\end{document}
